\documentclass[12pt,a4paper]{article}
\usepackage{times}
\usepackage[utf8]{inputenc}
\usepackage{amsmath}
\usepackage{amsfonts}
\usepackage{amssymb}
\usepackage[spanish]{babel}
\usepackage{bbm}
\usepackage{pdfpages}
\setlength{\parindent}{0cm} % Indent
\usepackage{geometry}[margins = 1in]
\geometry{letterpaper, top=20mm,bottom=20mm, left=25mm, right=25mm}
\newcommand \itsubsection{\titleformat{\subsection} {\normalfont \itshape}{\thesubsection}{1 em}{}}
\usepackage{natbib}
\usepackage{apalike}

% Gr�ficos en vez de Figura 
\usepackage{float} %Images float exactly where the code is
\usepackage{caption}
\usepackage{subcaption}
\usepackage{newfloat}
\usepackage{graphicx} % Required for including pictures
\DeclareFloatingEnvironment[name=Gr�fico]{graph}
\captionsetup[graph]{labelfont=bf, labelsep=newline}
\setcounter{secnumdepth}{0} % quitar la n�meraci�n de las secciones

% tabs caption 
\usepackage{threeparttable}
\usepackage{multicol}
\usepackage{multirow}
\usepackage{booktabs}
\captionsetup[table]{labelfont=bf, labelsep=newline}
\addto\captionsspanish{
\def\tablename{Tabla}
\def\listtablename{\'Indice de tablas}
}

% Defines columns for tables
\usepackage{array}
\newcolumntype{L}[1]{>{\raggedright\let\newline\\\arraybackslash\hspace{0pt}}m{#1}}
\newcolumntype{C}[1]{>{\centering\let\newline\\\arraybackslash\hspace{0pt}}m{#1}}
\newcolumntype{R}[1]{>{\raggedleft\let\newline\\\arraybackslash\hspace{0pt}}m{#1}}
\renewcommand{\baselinestretch}{1.2}


\title{Assignment 4 \\ Causal Inference }
\author{}
\date{}
\setcounter{secnumdepth}{0}
\begin{document}
\maketitle

\section{1}\textbf{Sean $ x_{1},\ x_{2} $ variables continuas y $ d_{1} $ una variable dummy.} 

		\begin{enumerate}[label=\alph*]
			
			
			\item.\textbf{En el modelo probit:}
			\begin{equation*}
			P\left(y=1\mid x_{1},x_{2}\right)=\Phi\left(\delta x_{1} + \gamma_{1} x_{2} + \gamma_{2} x^{2}_{2} + \gamma_{3} d_{1} + \gamma_{4} x_{2} d_{1} \right)
			\end{equation*}
			\textbf{Encuentre el efecto marginal de $ x_{2} $ y $ d_{1} $ en la probabilidad de respuesta. �Como se estimar�an estos efectos?} 
			
			
			\item.\textbf{En el modelo probit:}
			\begin{equation*}
			P\left(y=1\mid x_{1},x_{2},q\right)=\Phi\left(\delta x_{1} + \gamma_{1} x_{2} q \right)
			\end{equation*}
			\textbf{Donde $ q $ es una variable no observadaindependiente de $ x_{1} $ y $ x_{2} $, y que se distribuye Normal(0,1). Encuentre el efecto marginal de $ x_{2} $ en la probabilidad de respuesta, esto es $ \dfrac{\delta P(y=1 \mid x_{1},x_{2},q)}{\delta x_{2}} $, �C�mo se estimar�a este efecto parcial?}
			
			\item\textbf{ Basado en el modelo del literal anterior muestre que: }
			\begin{equation*}
			P\left(y=1\mid x_{1},x_{2}\right)=\Phi\left[\dfrac{\delta x_{i}}{\left(1+\gamma^{2}_{1}x^{2}_{2}\right)^{1/2}}\right]
			\end{equation*}
			
			\item. \textbf{Defina $ \rho_{1}=\gamma^{2}_{1} $. Pruebe la hip�tesis $ H_{0}:\rho_{1}=0 $}
		\end{enumerate}
	


\end{document}

%%%